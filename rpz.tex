%% Преамбула TeX-файла

% 1. Стиль и язык
\documentclass[utf8x, 14pt]{G7-32} % Стиль (по умолчанию будет 14pt)

% Остальные стандартные настройки убраны в preamble.inc.tex.
\input{common/preamble.inc}

\graphicspath{ {graphics/} }

% Принудительно отключим minted
\def\NoMinted{}

% Настройки листингов.
\ifPDFTeX
\input{common/listings.inc}
\else
\usepackage{local-minted}
\fi

% Полезные макросы листингов.
\input{common/macros.inc}

% Стиль титульного листа и заголовки
\input{tex/00-title}

% Для тестового заполнения текстом lorem ipsum
\usepackage{blindtext}

\begin{document}

\frontmatter % выключает нумерацию ВСЕГО; здесь начинаются ненумерованные главы: реферат, введение, глоссарий, сокращения и прочее.

\maketitle %создает титульную страницу

\begin{executors}
\personalSignature{Первый исполнитель}{ФИО}

\personalSignature{Второй исполнитель}{ФИО}
\end{executors} % Титульный лист

% \NormRefs % Нормативные ссылки 

% \include{tex/00-abstract}

\include{tex/contents} % Оглавление
\include{tex/acronyms} % Список сокращений и условных обозначений
\include{tex/introduction} % Введение

\mainmatter % это включает нумерацию глав и секций в документе ниже

\chapter{Общая часть}
\label{cha:common-part}


\chapter{Специальная часть}
\label{cha:special-part}


% TODO: название
\chapter{Охрана труда...}
\label{cha:bzd}
% TODO: название
\chapter{Организационно-экономический раздел}
\label{cha:economics}

\backmatter %% Здесь заканчивается нумерованная часть документа и начинаются ссылки и
            
\include{tex/conclusion}
\bibliographystyle{bibtex-styles/gost780u}
\bibliography{biblio/rpz}


\appendix   % Тут идут приложения

% \include{tex/90-appendix1}
% \include{tex/91-appendix2}

\end{document}

%%% Local Variables:
%%% mode: latex
%%% TeX-master: t
%%% End:
