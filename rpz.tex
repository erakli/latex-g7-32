%% Преамбула TeX-файла

% 1. Стиль и язык
\documentclass[utf8x, 14pt]{G7-32} % Стиль (по умолчанию будет 14pt)

% Остальные стандартные настройки убраны в preamble.inc.tex.
\input{tex/preamble.inc}

\graphicspath{ {graphics/} }

% Принудительно отключим minted
\def\NoMinted{}

% Настройки листингов.
\ifPDFTeX
\input{tex/listings.inc}
\else
\usepackage{local-minted}
\fi

% Полезные макросы листингов.
\input{tex/macros.inc}

% Стиль титульного листа и заголовки
\input{tex/00-title}

% Для тестового заполнения текстом lorem ipsum
\usepackage{blindtext}

\begin{document}

\frontmatter % выключает нумерацию ВСЕГО; здесь начинаются ненумерованные главы: реферат, введение, глоссарий, сокращения и прочее.

\maketitle %создает титульную страницу


\begin{executors}
\personalSignature{Первый исполнитель}{ФИО}

\personalSignature{Второй исполнитель}{ФИО}
\end{executors}


\listoffigures                         % Список рисунков
\listoftables                          % Список таблиц

%\NormRefs % Нормативные ссылки 
% Команды \breakingbeforechapters и \nonbreakingbeforechapters
% управляют разрывом страницы перед главами.
% По-умолчанию страница разрывается.

% \nobreakingbeforechapters
% \breakingbeforechapters

\include{tex/00-abstract}

\tableofcontents

\printnomenclature % Автоматический список сокращений

\include{tex/12-intro}

\mainmatter % это включает нумерацию глав и секций в документе ниже

\include{tex/20-analysis}
\include{tex/30-design}
\include{tex/40-impl}
\chapter{Экспериментальный раздел}
\label{cha:research}

В данном разделе проводятся вычислительные эксперименты.
А на рис.~\ref{fig:spire01} показана схема мыслительного процесса автора...

\begin{figure}
  \centering
%  \includegraphics[width=\textwidth]{graphics/svg/pic01.svg}
  \caption{Как страшно жить}
  \label{fig:spire01}
\end{figure}


%%% Local Variables:
%%% mode: latex
%%% TeX-master: "rpz"
%%% End:

\include{tex/60-economics}
\include{tex/70-bzd}

\backmatter %% Здесь заканчивается нумерованная часть документа и начинаются ссылки и
            
\include{tex/80-conclusion}%% заключение


% % Список литературы при помощи BibTeX
% Юзать так:
%
% pdflatex rpz
% bibtex rpz
% pdflatex rpz

\bibliographystyle{bibtex-styles/gost780u}
\bibliography{biblio/rpz}

%%% Local Variables: 
%%% mode: latex
%%% TeX-master: "rpz"
%%% End: 



\appendix   % Тут идут приложения

\include{tex/90-appendix1}
\include{tex/91-appendix2}

\end{document}

%%% Local Variables:
%%% mode: latex
%%% TeX-master: t
%%% End:
