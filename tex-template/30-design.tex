\chapter{Конструкторский раздел}
\label{cha:design}

В данном разделе проектируется новая всячина.

\section{Архитектура всячины}

\subsection{Протестируем подпункт}
\subsubsection{А теперь подподпункт}


\paragraph{Проверка} параграфа. Вроде работает.
\paragraph{Вторая проверка} параграфа. Опять работает.

Вот.

\begin{itemize}
\item Это список с <<палочками>>.
\item Хотя он и по ГОСТ, но\dots
\end{itemize}

\begin{enumerate}
\item  Для списка, начинающегося с заглавной буквы, лучше список с цифрами.
\end{enumerate}

Формула \eqref{F:F1} совершено бессмысленна.

%Кстати, при каких-то условиях <<удавалось>> получить двойный скобки вокруг номеров формул. Вопрос исследуется.

\begin{equation}
a= cb
\label{F:F1}
\end{equation}

А формула~\eqref{eq:fourierrow} имеет некоторый смысл.
Кроме этого она пытается иллюстрировать применение окружения \Code{eqndesc} которое размещает формулу совместно с её описанием.
Однако обратите внимание на нумерацию формул~\eqref{eq:fourierrow} и \eqref{F:F2}, попробуйте добавить \Code{[H]} к такой формуле.

\begin{eqndesc}
    \begin{equation}\label{eq:fourierrow}
        f(x) = \frac{a_0}{2} + \sum\limits_{k=1}^{+\infty} A_k\cos\left(k\frac{2\pi}{\tau}x+\theta_k\right)
    \end{equation}

    где $A_k$ "--- амплитуда  k-го гармонического колебания,\\
    $A_k$ "--- амплитуда $k$-го гармонического колебания,\\
    $ k\frac{2\pi}{\tau} = k\omega$ "--- круговая частота гармонического колебания,\\
    $\theta_k$ "--- начальная фаза $k$-го колебания.
\end{eqndesc}


Окружение \texttt{cases} опять работает (см. \eqref{F:F2}), спасибо И. Короткову за исправления..


\begin{equation}
a= \begin{cases}
 3x + 5y + z, \mbox{если хорошо} \\
 7x - 2y + 4z, \mbox{если плохо}\\
 -6x + 3y + 2z, \mbox{если совсем плохо}
\end{cases}
\label{F:F2}
\end{equation}

\section{Подсистема всякой ерунды}

Культурная вставка dot-файлов через утилиту dot2tex (рис.~\ref{fig:fig02}).

\begin{figure}
 \centering
% [width=0.5\textwidth] --- регулировка ширины картинки
% \includegraphics[width=.5\textwidth]{graphics/dot/cow2.dot}
 \caption{Рисунок}
 \label{fig:fig02}
\end{figure}


\subsection{Блок-схема всякой ерунды}

\subsubsection*{Кстати о заголовках}

У нас есть и \Code{subsubsection}. Только лучше её не нумеровать.

%%% Local Variables:
%%% mode: latex
%%% TeX-master: "rpz"
%%% End:
