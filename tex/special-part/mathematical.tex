\section{Математическая постановка задачи}

Будем рассматривать конечную группу однотипных беспилотников $ N = \{1, \dots, n\} $, где $ n $~-- число аппаратов.
Множество целей будем обозначать $ M = \{1, \dots, m\} $.
Любой беспилотник $ i $ имеет ограниченную дальность атаки $ d_{i}^{max} $.
Некоторая цель $ j \in M $ может быть атакована беспилотником $ i \in N $, если выполняется условие:
\begin{equation} \label{eq:ineq-distance}
    d_{ij} < d_{i}^{max} ,
\end{equation}
где $ d_{ij} $~-- расстояние между беспилотником и целью.
Всё множество возможных атак, которые может произвести аппарат $ i $, будем обозначать 
\begin{equation}
    \{a_0, a_1, \dots, a_m\} \in A_i ,
\end{equation}
причём в любой момент времени может производится только одна атака.
Назовём это множество \textit{профилем атаки}.
Под $ a_0 $ будет подразумеваться так называемая \textit{фиктивная} атака~-- выполняя это действие, беспилотник фактически ничего не делает.
Тогда $ A = A_1 \times \dots \times A_n $~-- множество всех профилей атак.
Обозначим $ R = \{r_1, \dots, r_n\} $, где $ r_i: A \rightarrow \mathbbm{R} $ вещественная функция выигрыша для беспилотника~$ i $.

% TODO перенести
Введём вероятность поражения цели $ p_{kill}(d_{ij}) $ как функцию расстояния между беспилотником и целью.
Если цель $ j $ в результате атаки $ a_j \in A_i $ была успешно уничтожена, то она удаляется из $ A $ и более не рассматривается как объект для атаки.

Для всех беспилотников вводится ограничение на коммуникацию.
Под этим будем понимать отсутствие какого либо информационного обмена между членами группы (такое условие накладывается, например, из соображений скрытности).

Обозначим задачу оптимального (в некотором смысле) распределения целей (и их последующей атаки) между беспилотниками $ N $.
Оптимальным будем полагать такое распределение, при котором по окончании выполнения миссии, будет нанесён максимальный урон противнику.
Так как непосредственное взаимодействие между беспилотниками исключается, каждый из них должен автономно принимать такие решения, чтобы сумма действий всех беспилотников приводила к максимальному ущербу.
Каждый беспилотник $ i $ должен стремиться максимизировать свою собственную функцию выигрыша $ r_i $ (вид этой функции мы определим позднее).

Очевидно, что функция выигрыша зависит от возможности уничтожения~-- чем выше шансы на поражение цели, тем более привлекательной должна быть цель для атаки конкретным беспилотником.
Будем считать, что на каждом беспилотнике выполняется абсолютно одинаковый алгоритм целераспределения, а любая информация, которой он обладает может быть получена только за счёт собственных средств наблюдения.

% добавить картинку
Рассмотрим следующую ситуацию. 
Допустим, что для некоторой цели $ t $ и некоторых беспилотников $ l, k $ одновременно выполняются условия
\begin{equation} \label{eq:ineq-distance-intersection}
    \begin{cases}
        d_{lt} < d_{l}^{max}, \\
        d_{kt} < d_{k}^{max}, \\
        d_{lt} \ne d_{kt}.
    \end{cases}
\end{equation}
В таком случае, атаку должен производить тот беспилотник, для которого $ r_i, \forall i \in \{l, k\} $ будет иметь б\'{о}льшее значение,  
\begin{equation}
    r_i < r_j, \quad
        \forall i, j \in \{l, k\} .
\end{equation}
% TODO можно ввести формальное определение зоны поражения
Нетрудно доказать, что такой алгоритм будет оптимальным для любой цели $ t $, для которой выполняются условия \eqref{eq:ineq-distance-intersection}.

Дополним описанный выше случай. 
Будем теперь рассматривать 2 цели, $ t_1, t_2 $, причём для $ t_1 $ справедливо \eqref{eq:ineq-distance-intersection}, а для $ t_2 $ введём новое условие:
\begin{equation}
    \begin{cases}
        d_{lt_2} < d_{l}^{max} , \\
        d_{kt_2} \ge d_{k}^{max} .
    \end{cases}
\end{equation}
В таком случае, беспилотник $ k $ будет обладать неполной информацией относительно решения, принимаемого беспилотником $ l $ (что следует из ограничения, наложенного на обмен информацией).
Последний, в свою очередь, может иметь максимальное значение $ r_l $ при атаке цели $ t_1 $.
Тогда, для беспилотника $ k $ может оказаться выгоднее не наносить атаку.

% TODO не нравится
Для того, чтобы разрешать подобные ситуации, введём следующее. 
Во-первых, необходимо сформировать такую функцию выигрыша $ r_i $, чтобы как можно меньшее количество целей оказалось не атакованными.
Во-вторых, будем рассматривать выбор цели как игру в нормальной форме.

% TODO здесь повторы пошли
Введём в рассмотрение конечную игру $ G = (N, A, R) $, где
\begin{itemize}
    \item $ N $ -- конечное множество игроков (БЛА),
    \item $ A $ -- конечное множество всех профилей атак,
    \item $ R $ -- конечное множество функций выигрыша.
\end{itemize}